\documentclass[11pt,a4paper]{article}
\usepackage[margin=2.5cm]{geometry}
\usepackage{amsmath,amssymb}
\usepackage{booktabs}
\usepackage{graphicx}
\usepackage{xcolor}
\usepackage{enumitem}
\usepackage{caption}
\usepackage{subcaption}
\usepackage{hyperref}
\usepackage{float}
\usepackage{natbib}

\title{Dual-Memory Norm Emergence Model:\\
Ablation Results and Cascade Mechanism Analysis}
\author{Preliminary Report}
\date{\today}

\begin{document}
\maketitle
\tableofcontents
\newpage

%% ════════════════════════════════════════════════════════════════
\section{Introduction}
\label{sec:intro}
%% ════════════════════════════════════════════════════════════════

How does a population of anonymous agents, starting from a perfectly symmetric 50-50 behavioural split and no shared history, arrive at a self-enforcing social norm---not merely a behavioural convention, but a shared rule that agents internalise, comply with, and enforce \citep{bicchieri2006grammar}? This report presents ablation and cascade analysis for a dual-memory model that addresses this question. The model equips each agent with two qualitatively distinct memory systems, grounded in separate empirical traditions, and unifies them through a single bridging variable: predictive confidence.

The first mechanism---\emph{experiential memory}---draws on the decisions-from-experience paradigm. \citet{hertwig2010decisions} argued that human decision-makers systematically rely on small, recent samples rather than exhaustive histories; computational models that best predict human behaviour in repeated market-entry games use effective windows of only five to six recent trials \citep{nevo2012bi}. Crucially, this window is not fixed. \citet{behrens2007learning} showed that the human brain continuously tracks environmental volatility and adjusts its learning rate accordingly: in volatile environments where prediction errors are frequent, learning rate increases---mathematically equivalent to shortening the effective memory window---so that recent observations receive greater weight; in stable environments, learning rate decreases, allowing longer integration over past experience. We operationalise this finding through a confidence-adaptive first-in-first-out (FIFO) buffer: predictive confidence~$C$ serves as the agent's estimate of environmental stability (the inverse of perceived volatility), and the effective window length is a monotonically increasing function of~$C$. Successful predictions raise~$C$ and expand the window, stabilising beliefs; prediction failures lower~$C$ and contract it, restoring sensitivity to change. The asymmetric update rule---confidence builds slowly through successful predictions but collapses rapidly after failures---is inspired by \citeauthor{slovic1993perceived}'s (\citeyear{slovic1993perceived}) observation that trust is easier to destroy than to create; we apply the same asymmetry to an agent's confidence in environmental predictability.

The second mechanism---\emph{normative memory}---operationalises norm internalisation as a drift-diffusion process. \citet{germar2014social} demonstrated experimentally that exposure to majority opinion alters the drift rate in perceptual decision-making, shifting subjects' evidence accumulation toward the majority-consistent response. \citet{germar2019learning} further showed that this perceptual shift persists after the social influence is removed, suggesting that repeated exposure to social consensus can produce a lasting cognitive bias. We interpret this persistence as evidence that social influence, under appropriate conditions, can give rise to internalised rules. We model the process as a signed two-boundary drift-diffusion model (DDM) in which observed social consistency accumulates as noisy evidence; when the accumulated evidence crosses a crystallisation threshold, a discrete normative rule forms.

The key architectural decision---that uncertainty promotes norm uptake while certainty promotes reliance on personal experience---is motivated by the broader principle that organisms increase social learning when individual learning is unreliable. \citet{rendell2010copy}, in their social learning strategies tournament, showed that the most successful strategies were those that copied when uncertain---that is, agents who relied on social information precisely when their own assessment of the environment was poor. This maps directly onto our confidence-gated architecture: the DDM drift rate scales with $(1-C)$, so low-confidence agents---those whose predictions of partner behaviour have been failing---accumulate normative evidence rapidly, while high-confidence agents rely predominantly on their experiential memory.

A related but distinct intuition comes from the dual-process literature. \citet{rand2014social} showed that under time pressure or cognitive load, people default to socially learned heuristics, while under deliberation they override these defaults. Our model captures an analogous asymmetry at the learning level rather than the execution level: just as cognitive load shifts behaviour toward social defaults in Rand's framework, low predictive confidence shifts an agent's learning dynamics toward faster uptake of social regularities. We do not claim a strict correspondence---Rand's heuristics are pre-existing defaults being retrieved, whereas our agents are forming new normative representations---but the directional logic is the same: uncertainty favours social over individual information.

Predictive confidence~$C$ thus serves as the single variable that bridges the two memory systems, determining whether an agent's learning is dominated by socially inferred rules or by individually accumulated experience.

The remainder of this report is organised as follows. Section~\ref{sec:model} specifies the model. Section~\ref{sec:ablation} presents a systematic ablation isolating each mechanism's contribution. Section~\ref{sec:cascade} traces the \emph{normative cascade}---the positive-feedback process by which minority norms dissolve and the population locks into consensus---through detailed tick-by-tick analysis. The cascade emerges from the interaction of both mechanisms and cannot be produced by either alone. Section~\ref{sec:enforcement} examines how enforcement pressure ($\Phi > 0$) reshapes the phase boundary, creates qualitatively distinct convergence pathways, and exhibits a non-monotonic effect on convergence. Section~\ref{sec:robustness} reports robustness checks.

%% ════════════════════════════════════════════════════════════════
\section{Model Description}
\label{sec:model}
%% ════════════════════════════════════════════════════════════════

\subsection{Environment and task}

$N$ agents (even) repeatedly play a symmetric pure coordination game with strategy set $S = \{A, B\}$. There is no payoff asymmetry---$A$ and $B$ are \emph{a priori} equivalent. Each tick proceeds through a synchronous six-stage pipeline.

\paragraph{Information structure.} At each tick, agents are randomly paired via uniform permutation. Each agent observes only its partner's chosen strategy (and optionally $V$ additional strategies sampled from other pairs). No agent ever observes any global statistic. All information is local.

\subsection{Agent state}

Each agent carries two memory systems, a confidence tracker, and derived action variables:

\begin{center}
\begin{tabular}{@{}lll@{}}
\toprule
Component & Variables & Role \\
\midrule
Experiential memory & FIFO buffer, $b_{\exp}$, $w$ & Track partner action frequencies \\
Confidence & $C \in [0,1]$ & Prediction accuracy tracker \\
Normative memory & $r$, $\sigma$, $a$, $e$ & Norm rule, strength, anomaly, DDM evidence \\
Derived & compliance $= \sigma^k$, $b_{\mathrm{eff}}$ & Action probability \\
\bottomrule
\end{tabular}
\end{center}

\subsection{Per-tick pipeline}

\begin{enumerate}[nosep]
\item \textbf{Stage 1: Pair and act.} Random permutation forms $N/2$ pairs. Each agent computes an effective belief $b_{\mathrm{eff}}$ and samples an action via probability matching ($P(\text{play } A) = b_{\mathrm{eff}}^A$). Each agent also makes a MAP prediction of its partner's action.

\item \textbf{Stage 2: Observe and update experiential memory.} The partner's action is pushed into a FIFO circular buffer (capacity $= w_{\max}$). The experiential belief is recomputed from the most recent $w$ entries:
\begin{equation}
b_{\exp}^A = \frac{\#A \text{ in last } w \text{ entries}}{w}
\label{eq:bexp}
\end{equation}
Additionally, $V$ strategies are sampled from other pairs for the normative observation set $O_i(t)$.

\item \textbf{Stage 3: Confidence update.} The agent compares its MAP prediction to the partner's actual action:
\begin{equation}
C \leftarrow \begin{cases}
C + \alpha(1 - C) & \text{if prediction correct (additive increase)} \\
C(1 - \beta) & \text{if prediction wrong (multiplicative decay)}
\end{cases}
\label{eq:confidence}
\end{equation}
The memory window is then recomputed: $w = w_{\mathrm{base}} + \lfloor C \cdot (w_{\max} - w_{\mathrm{base}}) \rfloor$. High confidence $\to$ long memory; low confidence $\to$ short memory.

\item \textbf{Stage 4: Normative update.} This stage branches based on crystallisation state.

\item \textbf{Stage 5: Enforcement.} Crystallised agents above a strength threshold ($\sigma > \theta_{\mathrm{enforce}}$) send a signal to partners who violated their norm. The signal enters the partner's DDM as a one-tick-delayed push: $\Phi (1-C) \gamma_{\mathrm{signal}} \cdot \mathrm{dir}$. (Analysis in Section~\ref{sec:enforcement}.)

\item \textbf{Stage 6: Metrics.} Read-only computation of population-level statistics.
\end{enumerate}

\subsection{Experiential memory: mean-field estimation}
\label{sec:exp}

Since partners are drawn by random pairing, each observation is an independent sample from the population's action distribution. The FIFO buffer stores $w \in [w_{\mathrm{base}}, w_{\max}]$ such samples, so $b_{\exp}^A$ (Eq.~\ref{eq:bexp}) is an unbiased but noisy estimator of the true population frequency $f_A$:
\begin{equation}
\mathbb{E}[b_{\exp}^A] = f_A, \qquad \mathrm{Var}(b_{\exp}^A) = \frac{f_A(1 - f_A)}{w}
\end{equation}

With $w \in [2, 6]$, this variance is substantial. Convergence under experiential memory alone is a \emph{diffusion} process: each agent independently drifts toward the majority, one noisy sample at a time.

\subsection{Normative memory: the DDM-crystallisation system}
\label{sec:norm}

Normative memory operates in two distinct modes, forming a state machine:

\begin{center}
\begin{tabular}{@{}c@{}}
\texttt{NO NORM} $\xrightarrow{|e| \geq \theta_{\mathrm{crystal}}}$ \texttt{HAS NORM}
$\xrightarrow[\sigma < \sigma_{\min}]{\text{crisis} \to \text{dissolve}}$
\texttt{NO NORM}
\end{tabular}
\end{center}

\paragraph{Pre-crystallisation (DDM).} An evidence accumulator $e$ integrates social observations:
\begin{equation}
e \leftarrow e + \underbrace{(1 - C) \cdot f_{\mathrm{diff}}}_{\text{confidence-gated drift}} + \underbrace{\mathcal{N}(0, \sigma_{\mathrm{noise}}^2)}_{\text{noise}}
\label{eq:ddm}
\end{equation}
where $f_{\mathrm{diff}} = (n_A - n_B)/|O_i|$ measures the signed consistency of the observation set $O_i(t)$. When $|e| \geq \theta_{\mathrm{crystal}}$, the agent \emph{crystallises} a norm: $r = A$ if $e > 0$, $r = B$ if $e < 0$. The norm strength is initialised to $\sigma_0$.

Key property: the drift is gated by $(1 - C)$. Low-confidence agents (uncertain about the environment) accumulate evidence faster and crystallise sooner.

At a symmetric 50-50 population ($f_{\mathrm{diff}} = 0$), the accumulator performs a pure random walk. The expected crystallisation time is $\theta_{\mathrm{crystal}}^2 / \sigma_{\mathrm{noise}}^2$. When the population is asymmetric (e.g., 70\% $A$), there is a positive drift and crystallisation occurs rapidly, predominantly toward the majority direction.

\paragraph{Post-crystallisation (norm maintenance).} Once crystallised, the agent monitors conformity and violations among its observations:
\begin{itemize}[nosep]
\item \textbf{Conformity} (observed action matches norm $r$): $\sigma$ increases via $\sigma \leftarrow \sigma + \alpha_\sigma(1 - \sigma)$.
\item \textbf{Violation} (observed action $\neq r$): anomaly counter $a$ increments.
\item \textbf{Crisis}: when $a \geq \theta_{\mathrm{crisis}}$, the norm weakens: $\sigma \leftarrow \lambda_{\mathrm{crisis}} \cdot \sigma$, and $a$ resets to 0.
\item \textbf{Dissolution}: if $\sigma < \sigma_{\min}$ after a crisis, the norm is abandoned. The agent returns to the DDM ($e = 0$) and may crystallise a new norm---potentially for the opposite strategy.
\end{itemize}

\subsection{Effective belief: the blending equation}
\label{sec:blend}

A crystallised agent's action probability blends the norm direction with the experiential estimate:
\begin{equation}
b_{\mathrm{eff}}^A = \underbrace{\sigma^k}_{\text{compliance}} \cdot b_{\mathrm{norm}}^A + (1 - \sigma^k) \cdot b_{\exp}^A
\label{eq:beff}
\end{equation}
where $b_{\mathrm{norm}}^A = 1$ if $r = A$ and $0$ if $r = B$, and $k$ is a compliance exponent. For an $A$-crystallised agent with $\sigma = 0.8$, $k = 2$: compliance $= 0.64$, so $b_{\mathrm{eff}}^A = 0.64 + 0.36 \cdot b_{\exp}^A$. Even if $b_{\exp}^A = 0.5$, the agent acts $A$ with probability $0.82$.

An uncrystallised agent simply uses $b_{\mathrm{eff}} = b_{\exp}$ (compliance $= 0$).

\subsection{Parameters}

\begin{table}[H]
\centering
\caption{Key model parameters and default values.}
\label{tab:params}
\small
\begin{tabular}{@{}llrl@{}}
\toprule
Symbol & Meaning & Default & Source \\
\midrule
\multicolumn{4}{@{}l}{\emph{Experiential layer}} \\
$N$ & Population size & 100 & --- \\
$\alpha$ & Confidence increase rate & 0.1 & Slovic 1993 \\
$\beta$ & Confidence decrease rate & 0.3 & Slovic 1993 \\
$C_0$ & Initial confidence & 0.5 & --- \\
$w_{\mathrm{base}}, w_{\max}$ & Memory window range & 2, 6 & Hertwig 2010 \\
\addlinespace
\multicolumn{4}{@{}l}{\emph{Normative layer}} \\
$\sigma_{\mathrm{noise}}$ & DDM noise & 0.1 & Germar 2014 \\
$\theta_{\mathrm{crystal}}$ & Crystallisation threshold & 3.0 & calibration \\
$\sigma_0$ & Initial norm strength & 0.8 & calibration \\
$\theta_{\mathrm{crisis}}$ & Crisis threshold (anomaly count) & 10 & calibration \\
$\lambda_{\mathrm{crisis}}$ & Crisis decay factor & 0.3 & calibration \\
$\sigma_{\min}$ & Dissolution threshold & 0.1 & calibration \\
$\alpha_\sigma$ & Strengthening rate & 0.005 & Will 2023 \\
$k$ & Compliance exponent & 2.0 & calibration \\
\addlinespace
\multicolumn{4}{@{}l}{\emph{Environment}} \\
$V$ & Additional observations/tick & 0 & --- \\
$\Phi$ & Enforcement gain & 0.0 & disabled \\
\bottomrule
\end{tabular}
\end{table}

\subsection{Initialisation}

All agents start identical: $b_{\exp} = (0.5, 0.5)$, $C = C_0$, $r = \text{none}$, $\sigma = 0$, $e = 0$. The population is perfectly symmetric with respect to $A$ and $B$.

%% ════════════════════════════════════════════════════════════════
\section{Ablation Experiment}
\label{sec:ablation}
%% ════════════════════════════════════════════════════════════════

\subsection{Experimental design}

To isolate each memory system's contribution, we ran a $3 \times 2$ factorial:

\begin{itemize}[nosep]
\item \textbf{Factor 1---Experiential memory level} (3 levels):
  \begin{itemize}[nosep]
    \item \emph{None}: $b_{\exp}$ reset to 0.5 after every tick (no individual learning).
    \item \emph{Fixed}: FIFO learning with fixed window $w = 5$.
    \item \emph{Dynamic}: FIFO learning with confidence-driven window $w \in [2, 6]$.
  \end{itemize}
\item \textbf{Factor 2---Normative memory}: OFF / ON.
\end{itemize}

Crossed with three population sizes $N \in \{20, 100, 500\}$. Each cell was run for 100 independent trials ($T = 3000$ ticks max).

\paragraph{Convergence criterion.} Behavioural majority $\geq 0.95$ sustained for 50 consecutive ticks.

\subsection{Results}

\begin{table}[H]
\centering
\caption{Convergence rate (out of 100 trials) and mean convergence tick.}
\label{tab:ablation}
\begin{tabular}{@{}ll crc crc crc@{}}
\toprule
& & \multicolumn{3}{c}{$N = 20$} & \multicolumn{3}{c}{$N = 100$} & \multicolumn{3}{c}{$N = 500$} \\
\cmidrule(lr){3-5} \cmidrule(lr){6-8} \cmidrule(lr){9-11}
Exp.\ Level & Norm & Rate & Tick & & Rate & Tick & & Rate & Tick & \\
\midrule
None    & OFF & 0/100  & ---   && 0/100  & ---    && 0/100  & ---    \\
None    & ON  & 0/100  & ---   && 0/100  & ---    && 0/100  & ---    \\
\addlinespace
Fixed   & OFF & 100 & 232  && 91 & 1125   && 24 & 2079   \\
Fixed   & ON  & 100 & 30   && 100 & 42    && 100 & 54    \\
\addlinespace
Dynamic & OFF & 100 & 76   && 100 & 231   && 100 & 467   \\
Dynamic & ON  & 100 & 25   && 100 & 37    && 100 & 43    \\
\bottomrule
\end{tabular}
\end{table}

\begin{table}[H]
\centering
\caption{Mean final majority fraction (all 100 trials, including non-converged).}
\label{tab:majority}
\begin{tabular}{@{}ll rrr@{}}
\toprule
Exp.\ Level & Norm & $N=20$ & $N=100$ & $N=500$ \\
\midrule
None  & OFF & 0.598 & 0.534 & 0.518 \\
None  & ON  & 0.632 & 0.615 & 0.616 \\
Fixed & OFF & 1.000 & 0.970 & 0.787 \\
Fixed & ON  & 1.000 & 1.000 & 1.000 \\
Dynamic & OFF & 1.000 & 0.993 & 0.957 \\
Dynamic & ON  & 1.000 & 1.000 & 1.000 \\
\bottomrule
\end{tabular}
\end{table}

\begin{table}[H]
\centering
\caption{Normative speedup ratio: mean convergence tick (Exp-only) / mean convergence tick (Exp+Norm).}
\label{tab:speedup}
\begin{tabular}{@{}l rrr@{}}
\toprule
Exp.\ Level & $N = 20$ & $N = 100$ & $N = 500$ \\
\midrule
Fixed ($w{=}5$) & $7.7\times$ & $26.5\times$ & $38.7\times$ \\
Dynamic $[2,6]$ & $3.0\times$ & $6.2\times$ & $10.8\times$ \\
\bottomrule
\end{tabular}
\end{table}

\subsection{Observations}

\begin{enumerate}
\item \textbf{Experiential memory is necessary.} Without it (None rows), convergence is 0/100 at all $N$, regardless of normative memory. Adding norms to frozen beliefs raises final majority from $\sim$0.53 to $\sim$0.62 (Table~\ref{tab:majority})---a modest improvement from crystallisation, but far below the 0.95 threshold. Normative memory cannot create a pattern from noise alone.

The mechanism is symmetric cancellation. With $b_{\exp}$ frozen at $0.5$, actions are coin flips, so the DDM performs a pure random walk and crystallises roughly equal numbers of $A$- and $B$-norms (${\sim}50$ vs.\ ${\sim}50$). An $A$-norm agent acts $A$ with $b_{\mathrm{eff}}^A = 0.82$; a $B$-norm agent acts $A$ with $b_{\mathrm{eff}}^A = 0.18$. These biases cancel at the population level: $\mathrm{fraction}_A \approx 0.50 \cdot 0.82 + 0.50 \cdot 0.18 = 0.50$. The resulting violation-rate gap between sides is negligible (${\sim}51\%$ vs.\ ${\sim}49\%$), so both norms reach crisis and dissolve at nearly the same rate---the one-way ratchet that drives the cascade in the full model (Section~\ref{sec:loop}) is effectively disabled. The modest improvement to $0.62$ arises from weak statistical drift accumulated over many crystallise--dissolve--recrystallise cycles, but this drift can never self-amplify to $0.95$ because $b_{\exp}$ is pinned at $0.5$ and the positive-feedback loop through experiential memory is broken.

\item \textbf{Experiential learning alone scales poorly.} Fixed-window convergence degrades from 100\% ($N{=}20$) to 24\% ($N{=}500$), and the mean tick grows roughly linearly with $N$. Dynamic window helps (100\% at all $N$), but the tick still grows from 76 to 467.

\item \textbf{With both systems, convergence is fast, complete, and $N$-robust.} The mean tick only grows from 25 to 43 as $N$ increases 25-fold (Table~\ref{tab:ablation}). Convergence rate is 100\% everywhere. Final majority is always 1.0.

\item \textbf{The speedup grows with $N$} (Table~\ref{tab:speedup}). This is not a constant-factor improvement---it suggests a qualitative change in the convergence mechanism.

\item \textbf{Normative memory compensates for weak experiential learning.} Fixed+Norm and Dynamic+Norm converge at nearly identical speeds ($\sim$10 ticks apart at all $N$). Once the normative cascade fires, the quality of the experiential estimator barely matters.
\end{enumerate}

%% ════════════════════════════════════════════════════════════════
\section{Deep Analysis: The Normative Cascade}
\label{sec:cascade}
%% ════════════════════════════════════════════════════════════════

The ablation shows \emph{what} each system contributes. This section traces \emph{how}: we tracked every agent's internal state tick by tick in a single diagnostic trial ($N = 100$, Dynamic+Norm, seed = 42).

\subsection{Five phases of norm emergence}

Figure~\ref{fig:anatomy} shows the number of agents in each crystallisation state over time.

\begin{figure}[H]
    \centering
    \includegraphics[height=0.65\textheight,angle=90,keepaspectratio]{figures/fig2_cascade_anatomy.png}
    \caption{Agent group dynamics: crystallised-$A$ (blue), crystallised-$B$ (red), uncrystallised (grey dashed). Vertical dotted lines mark phase boundaries.}
    \label{fig:anatomy}
\end{figure}

\begin{enumerate}[nosep]
\item \textbf{Symmetric random} (tick 1--7). All agents uncrystallised. Actions are coin flips based on $b_{\exp} \approx 0.5$.

\item \textbf{Symmetric crystallisation} (tick 7--30). DDM random walks cross $\pm\theta_{\mathrm{crystal}}$. Agents crystallise roughly evenly: 39\,$A$ vs.\ 38\,$B$ at tick 30. Both sides begin accumulating anomalies from each other. $b_{\exp}$ has barely moved from 0.5.

\item \textbf{Tipping} (tick 30--55). Experiential memory slowly shifts $b_{\exp}$ from 0.50 to $\sim$0.58. The slight $A$-majority means $B$-norms face more violations than $A$-norms. $B$-norms begin dissolving; $A$-norms hold.

\item \textbf{Cascade} (tick 55--72). The $B$-norm count collapses from 27 to 0 in $\sim$15 ticks. Each dissolution feeds the next (see Section~\ref{sec:loop}).

\item \textbf{Lock-in} (tick 72+). All agents are $A$-crystallised. Norm strength recovers, confidence rises to 1.0, $b_{\exp} \to 1.0$.
\end{enumerate}

\subsection{Both vs.\ Experiential-only: same seed}

Figure~\ref{fig:comparison} overlays the same seed under both conditions. The trajectories are virtually identical until tick 50; the divergence corresponds exactly to the cascade phase.

\begin{figure}[H]
    \centering
    \includegraphics[width=0.85\textwidth]{figures/fig1_both_vs_exponly.png}
    \caption{Fraction playing $A$: Dual Memory (solid blue) vs.\ Experiential Only (dashed orange), same seed. Shaded region marks the cascade.}
    \label{fig:comparison}
\end{figure}

\subsection{Signal amplification}

Figure~\ref{fig:amplifier} reveals the mechanism behind the divergence. $A$-norm agents act $A$ with $\sim$85\% probability even when $b_{\exp} \approx 0.55$---the blending equation (Eq.~\ref{eq:beff}) converts a weak field signal into strong behavioural commitment. $B$-norm agents act $A$ with only $\sim$12\%. This bimodal output drives the population fraction far more aggressively than $b_{\exp}$ alone could.

\begin{figure}[H]
    \centering
    \includegraphics[width=0.85\textwidth]{figures/fig4_amplifier.png}
    \caption{Per-group effective belief $b_{\mathrm{eff}}^A$. Blue: $A$-norm. Red: $B$-norm. Grey dashed: uncrystallised. Green dotted: population mean $b_{\exp}$.}
    \label{fig:amplifier}
\end{figure}

\subsection{The feedback mechanism}

Figure~\ref{fig:feedback} traces four coupled variables through the cascade:

\begin{figure}[H]
    \centering
    \includegraphics[width=\textwidth]{figures/fig3_feedback_loop.png}
    \caption{(a)~Norm strength $\sigma$: $B$-norms decay faster. (b)~Anomaly counts: $B$-norms approach the crisis threshold while $A$-norms decline. (c)~$b_{\exp}$ tracks the field; $b_{\mathrm{eff}}$ leads it. (d)~Dissolution events (red) and new $A$-crystallisations (blue) cluster in tick 40--70.}
    \label{fig:feedback}
\end{figure}

\subsection{The one-way ratchet}

We tracked every norm-state transition for all 100 agents. Among agents that crystallised to $B$, dissolved, and subsequently re-crystallised:
\begin{itemize}[nosep]
\item 44 agents followed $B \to \text{dissolved} \to A$.
\item \textbf{0 agents} followed $B \to \text{dissolved} \to B$.
\end{itemize}

Figure~\ref{fig:lifecycle} visualises these lifecycles. The mean uncrystallised interval is 11.2 ticks---once dissolved, the agent sees a strongly $A$-biased environment and crystallises to $A$ rapidly.

\begin{figure}[H]
    \centering
    \includegraphics[width=0.85\textwidth]{figures/fig5_agent_lifecycle.png}
    \caption{Lifecycle of 44 agents that transitioned $B \to \text{dissolved} \to A$. Red = $B$-crystallised, grey = uncrystallised, blue = $A$-crystallised. Black dots mark dissolution.}
    \label{fig:lifecycle}
\end{figure}

\subsection{The dual feedback loop}
\label{sec:loop}

The cascade arises from two interlocking positive-feedback loops.

\paragraph{Loop 1: Differential erosion (slow, tick 30--55).} When $n_A > n_B$ among crystallised agents, a $B$-norm agent's partner is more likely to play $A$ (a norm violation). Therefore $B$-norms accumulate anomalies faster, reach crisis sooner, and experience more $\sigma$-decay. As $B$-agents' compliance drops (Eq.~\ref{eq:beff}), they act less distinctly $B$, which makes the environment even more $A$-biased:
\begin{equation}
n_A > n_B \;\Longrightarrow\; \text{anomaly rate}(B) > \text{anomaly rate}(A) \;\Longrightarrow\; \sigma_B \downarrow\; \;\Longrightarrow\; \text{environment more } A\text{-biased}
\end{equation}

\paragraph{Loop 2: Dissolution cascade (fast, tick 55--72).} When $\sigma_B$ falls below $\sigma_{\min}$, the agent dissolves and re-enters the DDM. The DDM drift $(1 - C) \cdot f_{\mathrm{diff}}$ is now strongly positive (the environment is majority-$A$), so the agent re-crystallises to $A$ within $\sim$10 ticks. Each $B \to A$ conversion increases $n_A$, accelerating both loops:
\begin{equation}
\frac{d\, n_B}{dt} \;\propto\; -\,n_B \;\cdot\; p_A(n_B), \qquad \frac{\partial\, p_A}{\partial\, n_B} < 0
\end{equation}
(This is a qualitative continuous approximation; the actual dynamics proceed through discrete crisis events---anomaly accumulation to $\theta_{\mathrm{crisis}}$, multiplicative $\sigma$-decay, and dissolution when $\sigma < \sigma_{\min}$.)

\noindent The result is \emph{exponential acceleration}: each dissolution makes the next faster (Figure~\ref{fig:causal}d).

\begin{figure}[H]
    \centering
    \includegraphics[width=\textwidth]{figures/fig6_causal_chain.png}
    \caption{The causal chain. (a)~Field asymmetry drives crystallisation asymmetry. (b)~$\sigma_A - \sigma_B$ diverges. (c)~Cumulative dissolutions track cumulative $A$-crystallisations. (d)~$B$-norm count collapses with accelerating rate.}
    \label{fig:causal}
\end{figure}

\paragraph{Why this is a phase transition, not diffusion.}
\begin{itemize}[nosep]
\item \textbf{Diffusion} (experiential only): each agent independently drifts toward the majority at rate $\sim 1/w$. Convergence time scales as $O(N)$ because the signal must propagate one-hop-at-a-time through random pairings.
\item \textbf{Phase transition} (with norms): once a critical asymmetry is reached ($\sim$55--60\% crystallised to one side), the cascade is self-sustaining and completes in a time determined by the anomaly-to-crisis-to-dissolution pathway---\emph{not} by $N$.
\end{itemize}

This structural difference explains the scaling results in Table~\ref{tab:speedup}: the experiential convergence time grows linearly with $N$, while the cascade time remains approximately constant, so the speedup ratio increases with $N$.

%% ════════════════════════════════════════════════════════════════
\section{Enforcement Pressure and the Phase Boundary}
\label{sec:enforcement}
%% ════════════════════════════════════════════════════════════════

The preceding analysis established the normative cascade mechanism at $\Phi = 0$ (no enforcement). We now ask three questions about enforcement pressure:
\begin{enumerate}[nosep]
\item Does $\Phi$ shift the phase transition boundary? (known at $\theta \approx 7$ for $\Phi = 0$)
\item Does $\Phi$ create qualitatively different convergence pathways?
\item Is the effect of $\Phi$ monotonic, or does an optimal enforcement level exist?
\end{enumerate}

\subsection{Method}

All experiments use the full-model configuration ($w_{\mathrm{base}} = 2$, $w_{\max} = 6$, $V = 0$, normative enabled) with default crystallisation parameters ($\sigma_0 = 0.8$, $k = 2$, $\theta_{\mathrm{crisis}} = 10$). Only $\theta_{\mathrm{crystal}}$ and $\Phi$ are varied. Each condition was run for 30 independent trials ($T = 3000$). Beyond standard convergence metrics, we tracked six novel per-trial pathway metrics via a manual tick loop with transition detection:

\begin{itemize}[nosep]
\item \textbf{Total dissolutions}---number of norm dissolution events across all ticks.
\item \textbf{Total enforcements}---cumulative enforcement signals sent.
\item \textbf{Recrystallisation direction}---among dissolved agents that re-crystallised, how many switched norm (\emph{flip}) versus returned to the same norm (\emph{same}).
\item \textbf{Peak norm split}---$\max_t [\max(n_A, n_B) / (n_A + n_B)]$ among crystallised agents (worst polarisation).
\item \textbf{Time in churn}---ticks where both dissolutions and new crystallisations occurred simultaneously.
\end{itemize}

\subsection{Phase boundary shift}
\label{sec:boundary}

Table~\ref{tab:phase_boundary} shows convergence rates across a $\theta_{\mathrm{crystal}} \times \Phi$ grid (900 total runs).

\begin{table}[H]
\centering
\caption{Convergence rate (\%, 30 trials each). The phase boundary---where convergence drops below ${\sim}50\%$---shifts depending on $\Phi$.}
\label{tab:phase_boundary}
\begin{tabular}{@{}r rrrrr@{}}
\toprule
$\theta_{\mathrm{crystal}}$ & $\Phi{=}0$ & $\Phi{=}0.5$ & $\Phi{=}1$ & $\Phi{=}2$ & $\Phi{=}5$ \\
\midrule
3  & 100 & 100 & 100 & 100 & 100 \\
5  & 93  & 93  & 97  & 80  & 50 \\
7  & 37  & 50  & 60  & 67  & 73 \\
9  & 27  & 7   & 13  & 40  & 33 \\
12 & 3   & 7   & 7   & 7   & 10 \\
15 & 3   & 0   & 0   & 3   & 3 \\
\bottomrule
\end{tabular}
\end{table}

Two effects are visible:

\begin{enumerate}
\item \textbf{Above the baseline boundary ($\theta \geq 7$): enforcement helps.} At $\theta = 7$, convergence nearly doubles from 37\% ($\Phi = 0$) to 73\% ($\Phi = 5$). The mechanism is straightforward: at high crystallisation thresholds, many agents remain pre-crystallised for long periods. Enforcement signals deliver a push $\Phi(1-C)\gamma_{\mathrm{signal}} \cdot \mathrm{dir}$ into the DDM accumulator, accelerating crystallisation toward the direction of the enforcing (already-crystallised) agent. Since the majority-side crystallised agents are more numerous, the net signal push is majority-directed.

\item \textbf{Below the baseline boundary ($\theta \leq 5$): enforcement hurts.} At $\theta = 5$, convergence drops from 93\% ($\Phi = 0$) to 50\% ($\Phi = 5$). With a low crystallisation barrier, most agents crystallise quickly---but approximately half crystallise to each side (the population is still near 50-50 when most DDM accumulators cross the threshold). Once both sides are crystallised, enforcement between agents holding opposing norms creates \emph{mutual violations}: an $A$-norm agent enforces $A$ on a $B$-norm partner, but the partner is post-crystallised, so the enforcement is wasted and the violation counts as anomaly (DD-7). This accelerates crisis and dissolution on both sides, creating turbulence that prevents the cascade from completing.
\end{enumerate}

Table~\ref{tab:phase_dissolutions} confirms this mechanism: dissolutions increase with $\Phi$ at every $\theta$ level.

\begin{table}[H]
\centering
\caption{Mean dissolutions per trial.}
\label{tab:phase_dissolutions}
\begin{tabular}{@{}r rrrrr@{}}
\toprule
$\theta_{\mathrm{crystal}}$ & $\Phi{=}0$ & $\Phi{=}0.5$ & $\Phi{=}1$ & $\Phi{=}2$ & $\Phi{=}5$ \\
\midrule
3  & 15.2 & 16.2 & 16.8 & 21.6 & 27.6 \\
5  & 5.0  & 3.9  & 3.3  & 5.9  & 20.8 \\
7  & 1.2  & 2.3  & 1.7  & 2.9  & 8.4 \\
9  & 0.6  & 1.1  & 0.9  & 0.3  & 2.4 \\
12 & 0.4  & 0.5  & 0.1  & 0.4  & 0.2 \\
15 & 0.0  & 0.0  & 0.0  & 0.1  & 0.0 \\
\bottomrule
\end{tabular}
\end{table}

The $\theta = 3$ row is instructive: even though convergence is 100\% at all $\Phi$, the number of dissolutions rises from 15 to 28. Enforcement creates more turbulence, but the low crystallisation barrier allows rapid re-crystallisation, so the cascade still completes.

\subsection{Convergence pathways}
\label{sec:pathways}

To distinguish pathway types, we examined detailed metrics at two representative thresholds ($\theta = 3$ and $\theta = 7$) across $\Phi$ levels (300 runs; Table~\ref{tab:pathways}).

\begin{table}[H]
\centering
\caption{Pathway metrics (means over 30 trials). $\mu_{\mathrm{diss}}$: dissolutions. $\mu_{\mathrm{enf}}$: enforcements. $\mu_{\mathrm{churn}}$: ticks with simultaneous dissolutions and crystallisations. $\mu_{\mathrm{flip}}$/$\mu_{\mathrm{same}}$: recrystallisations switching/returning to same norm.}
\label{tab:pathways}
\small
\begin{tabular}{@{}rl rrrrrrr@{}}
\toprule
$\theta$ & $\Phi$ & Conv\% & $\mu_{\mathrm{diss}}$ & $\mu_{\mathrm{enf}}$ & $\mu_{\mathrm{churn}}$ & $\mu_{\mathrm{flip}}$ & $\mu_{\mathrm{same}}$ & $\mu_{\mathrm{tick}}$ \\
\midrule
3  & 0   & 100 & 12.3 & 0     & 4.0 & 9.9  & 0.0 & 89 \\
3  & 0.5 & 100 & 15.4 & 521   & 1.9 & 13.6 & 0.1 & 99 \\
3  & 1.0 & 100 & 15.7 & 477   & 2.0 & 13.0 & 1.1 & 93 \\
3  & 2.0 & 100 & 21.0 & 583   & 2.8 & 18.7 & 0.0 & 98 \\
3  & 5.0 & 100 & 33.2 & 761   & 4.8 & 27.9 & 2.2 & 107 \\
\addlinespace
7  & 0   & 53  & 2.0  & 0     & 1.5 & 0.7  & 0.3 & 103 \\
7  & 0.5 & 60  & 1.8  & 162   & 0.8 & 0.0  & 0.0 & 104 \\
7  & 1.0 & 67  & 1.3  & 134   & 0.4 & 0.0  & 0.0 & 99 \\
7  & 2.0 & 70  & 2.2  & 159   & 0.5 & 0.3  & 0.0 & 102 \\
7  & 5.0 & 60  & 3.8  & 172   & 0.4 & 0.6  & 0.0 & 95 \\
\bottomrule
\end{tabular}
\end{table}

Three qualitative pathway types emerge:

\paragraph{Clean cascade ($\Phi = 0$, low $\theta$).} The mechanism described in Section~\ref{sec:cascade}: agents crystallise to both sides, differential erosion breaks symmetry, minority norms dissolve and re-crystallise to the majority. The 9.9 flips and 0 same-direction recrystallisations confirm the one-way ratchet. Moderate churn (4.0 ticks) reflects the dissolution--recrystallisation wave.

\paragraph{Enforcement-assisted ($0 < \Phi \leq 2$, high $\theta$).} Enforcement signals accelerate DDM evidence accumulation in pre-crystallised agents, nudging them toward the majority norm. Dissolutions stay low ($\leq 2.2$), churn decreases from 1.5 to 0.4--0.5 ticks, and recrystallisation events nearly vanish---the cascade is smoother because fewer agents need to dissolve and switch. Convergence rate improves from 53\% to 67--70\%.

\paragraph{Enforcement-turbulent ($\Phi \geq 5$, low $\theta$).} High enforcement pressure generates 761 enforcement signals and 33.2 dissolutions per trial---nearly triple the no-enforcement baseline. Churn increases to 4.8 ticks. Yet convergence remains 100\%: the low crystallisation barrier ensures rapid re-crystallisation, and the one-way ratchet ($\mu_{\mathrm{flip}} = 27.9$, $\mu_{\mathrm{same}} = 2.2$) is maintained despite the turbulence. The cost is ${\sim}20\%$ slower convergence (107 vs.\ 89 ticks).

\paragraph{Key observation.} The one-way ratchet identified in Section~\ref{sec:cascade} is robust to enforcement. Even at $\Phi = 5$ (the highest tested), the flip-to-same ratio remains $>$10:1 at $\theta = 3$---dissolved agents overwhelmingly re-crystallise toward the majority. Enforcement creates more dissolution events but does not disrupt the directional bias of the cascade.

\subsection{Non-monotonic enforcement effect}
\label{sec:nonmonotonic}

To search for an optimal $\Phi$, we swept a fine grid of 11 values at three crystallisation thresholds (990 runs; Table~\ref{tab:nonmonotonic}).

\begin{table}[H]
\centering
\caption{Convergence rate (\%) and mean dissolutions across fine $\Phi$ grid (30 trials each).}
\label{tab:nonmonotonic}
\small
\begin{tabular}{@{}r rr rr rr@{}}
\toprule
& \multicolumn{2}{c}{$\theta = 3$} & \multicolumn{2}{c}{$\theta = 5$} & \multicolumn{2}{c}{$\theta = 7$} \\
\cmidrule(lr){2-3} \cmidrule(lr){4-5} \cmidrule(lr){6-7}
$\Phi$ & Rate & Diss. & Rate & Diss. & Rate & Diss. \\
\midrule
0   & 100 & 12.4 & 90  & 4.0  & 40  & 2.6 \\
0.1 & 100 & 11.6 & 87  & 3.2  & 47  & 1.2 \\
0.2 & 100 & 14.2 & 87  & 4.9  & 47  & 1.4 \\
0.3 & 100 & 10.5 & 87  & 4.1  & 77  & 2.4 \\
0.5 & 100 & 10.1 & 83  & 6.0  & 63  & 1.2 \\
0.7 & 100 & 15.4 & 90  & 4.7  & 60  & 2.2 \\
1.0 & 100 & 19.4 & 87  & 3.9  & 57  & 3.4 \\
1.5 & 100 & 21.2 & 73  & 5.1  & 53  & 3.8 \\
2.0 & 100 & 34.0 & 63  & 10.2 & 77  & 1.4 \\
3.0 & 100 & 37.8 & 53  & 17.8 & 80  & 2.9 \\
5.0 & 100 & 34.2 & 50  & 20.8 & 63  & 5.7 \\
\bottomrule
\end{tabular}
\end{table}

The three thresholds show qualitatively different $\Phi$-response profiles:

\begin{enumerate}
\item \textbf{$\theta = 3$ (below boundary): immune to $\Phi$.} Convergence rate is 100\% everywhere. Dissolutions rise monotonically from 12 to 38 as $\Phi$ increases, but the low barrier ensures re-crystallisation always completes the cascade.

\item \textbf{$\theta = 5$ (near boundary): monotonic degradation.} Convergence declines from 90\% to 50\% as $\Phi$ increases. Most agents crystallise quickly at $\theta = 5$, so enforcement primarily creates mutual violations between opposing crystallised agents, destabilising both sides (dissolutions rise from 4 to 21).

\item \textbf{$\theta = 7$ (at boundary): non-monotonic with possible bimodality.} Two local peaks appear: $\Phi \approx 0.3$ (77\%) and $\Phi \approx 3$ (80\%), with a valley at $\Phi \approx 1$--$1.5$ (53--57\%) and a decline at $\Phi = 5$ (63\%). While the 30-trial sample sizes preclude strong statistical claims about the exact peak locations, the overall pattern---improvement at both low and high $\Phi$ relative to the intermediate range---is consistent across the independently seeded analyses (compare Table~\ref{tab:phase_boundary}: $\theta = 7$ also shows monotonic improvement from 37\% to 73\%).
\end{enumerate}

A tentative interpretation of the bimodal pattern at $\theta = 7$: low $\Phi$ provides just enough signal push to tip marginal DDM accumulators over the crystallisation threshold without destabilising existing norms; intermediate $\Phi$ is strong enough to trigger enforcement-driven violations among the few early-crystallised agents but too weak to rapidly crystallise the remaining pre-crystallised majority; high $\Phi$ overwhelms both effects, converting nearly all pre-crystallised agents before mutual destabilisation can accumulate. This hypothesis requires larger sample sizes and agent-level trajectory analysis to confirm.

\subsection{Interpretation: two faces of enforcement}
\label{sec:two_faces}

The results reveal that enforcement operates through two competing channels:

\paragraph{Beneficial channel: DDM acceleration.} Enforcement signals deliver a push $\Phi(1-C)\gamma_{\mathrm{signal}} \cdot \mathrm{dir}$ into pre-crystallised agents' evidence accumulators. When the enforcing agent holds the majority norm, this accelerates crystallisation in the majority direction. This channel dominates when many agents remain pre-crystallised---i.e., when $\theta_{\mathrm{crystal}}$ is high.

\paragraph{Harmful channel: mutual destabilisation.} When two agents hold opposing crystallised norms, enforcement triggers mutual violations (DD-7: enforcement against a post-crystallised partner is wasted, and the violation counts as anomaly). This accelerates crisis on both sides, increasing dissolution and potentially creating churn cycles. This channel dominates when most agents are already crystallised---i.e., when $\theta_{\mathrm{crystal}}$ is low and agents crystallise early into a roughly symmetric split.

\paragraph{The balance.} The relative strength of these channels depends on the fraction of agents that are pre-crystallised when enforcement begins. At $\theta = 7$, the DDM barrier is high enough that the enforcement-beneficial window is large---many agents are still accumulating evidence when the first crystallised agents begin enforcing. At $\theta = 5$, the barrier is low enough that by the time enforcement becomes active (requiring $\sigma > \theta_{\mathrm{enforce}} = 0.7$), most agents have already crystallised to both sides, and enforcement primarily creates cross-norm destabilisation.

This mechanistic account also explains the high dissolution rate at $\theta = 3$ even at $\Phi = 0$ (12.4 dissolutions): with a very low barrier, agents crystallise before any behavioural asymmetry develops, ensuring roughly equal numbers on each side. The subsequent cascade must dissolve the entire minority side---a process that enforcement amplifies but cannot disrupt, because the one-way ratchet is maintained by the experiential memory signal that grows stronger as the cascade progresses.

%% ════════════════════════════════════════════════════════════════
\section{Robustness}
\label{sec:robustness}
%% ════════════════════════════════════════════════════════════════

\subsection{Cross-seed consistency (30 seeds)}

\begin{table}[H]
\centering
\caption{Cascade timing across 30 seeds ($N = 100$, Dynamic+Norm).}
\label{tab:robustness}
\begin{tabular}{@{}l rrr@{}}
\toprule
Metric & Mean & Median & Range \\
\midrule
Symmetry breaks (last tick $|n_A - n_B| \leq 2$) & 10.4 & 7.5 & [0, 38] \\
Cascade completes (minority norm $= 0$) & 41.5 & 43.5 & [10, 79] \\
\textbf{Cascade duration} & \textbf{31.1} & \textbf{33.5} & \textbf{[4, 45]} \\
Behavioural convergence tick & 34.5 & 33.5 & [17, 70] \\
\bottomrule
\end{tabular}
\end{table}

The cascade duration is remarkably stable (median 33.5 ticks). Variability comes from the symmetry-breaking phase, not the cascade itself: the earliest trials happen to get an asymmetric initial draw, while the latest take $\sim$38 ticks to break symmetry.

\subsection{Population-size invariance}

From Table~\ref{tab:ablation}, Dynamic+Norm converges in 25 ticks at $N{=}20$, 37 at $N{=}100$, and 43 at $N{=}500$. A 25-fold increase in population adds only 18 ticks. The cascade mechanism---driven by local anomaly rates, not global diffusion---is inherently $N$-insensitive.

\subsection{Symmetry between strategies}

Of the 30 seeds, 16 converged to $A$ and 14 to $B$---consistent with the model's complete symmetry. The winning side is determined by stochastic fluctuations during the symmetry-breaking phase.

%% ════════════════════════════════════════════════════════════════
\section{Summary}
\label{sec:summary}
%% ════════════════════════════════════════════════════════════════

\begin{enumerate}
\item \textbf{Experiential memory} provides a noisy local estimate of the population behavioural frequency---a stochastic mean-field sensor. It can produce convergence alone, but slowly and with $O(N)$ scaling.

\item \textbf{Normative memory} acts as a signal amplifier: crystallisation converts a weak experiential signal ($b_{\exp} \approx 0.55$) into strong behavioural commitment ($b_{\mathrm{eff}} \approx 0.85$), and the dissolution-recrystallisation cycle ensures that amplification is unidirectional.

\item The amplification creates two interlocking positive-feedback loops (differential erosion $+$ dissolution cascade) that produce a \textbf{phase transition} from symmetric disorder to full consensus in $\sim$30 ticks, largely independent of $N$.

\item Neither system alone is sufficient: without experiential memory, the amplifier has no signal; without normative memory, the signal diffuses slowly.

\item \textbf{Enforcement pressure ($\Phi > 0$) has a dual nature.} It accelerates crystallisation in pre-crystallised agents (beneficial) but creates mutual destabilisation between agents holding opposing norms (harmful). The balance depends on $\theta_{\mathrm{crystal}}$: enforcement helps above the phase boundary ($\theta \geq 7$, convergence 37\% $\to$ 73\%) and hurts below it ($\theta = 5$, convergence 93\% $\to$ 50\%). The one-way ratchet is robust to enforcement---dissolved agents overwhelmingly re-crystallise toward the majority even at high $\Phi$.

\item All of this emerges from purely local interactions---no agent ever observes the global state.
\end{enumerate}

%% ════════════════════════════════════════════════════════════════
\bibliographystyle{apalike}
\begin{thebibliography}{99}

\bibitem[Behrens et al.(2007)]{behrens2007learning}
Behrens, T.~E.~J., Woolrich, M.~W., Walton, M.~E., \& Rushworth, M.~F.~S. (2007).
Learning the value of information in an uncertain world.
\textit{Nature Neuroscience}, 10(9), 1214--1221.

\bibitem[Bicchieri(2006)]{bicchieri2006grammar}
Bicchieri, C. (2006).
\textit{The Grammar of Society: The Nature and Dynamics of Social Norms}.
Cambridge University Press.

\bibitem[Germar et al.(2014)]{germar2014social}
Germar, M., Schlemmer, A., Krug, K., Voss, A., \& Mojzisch, A. (2014).
Social influence and perceptual decision making: A diffusion model analysis.
\textit{Personality and Social Psychology Bulletin}, 40(2), 217--231.

\bibitem[Germar \& Mojzisch(2019)]{germar2019learning}
Germar, M., \& Mojzisch, A. (2019).
Learning of social norms can lead to a persistent perceptual bias: A diffusion model approach.
\textit{Journal of Experimental Social Psychology}, 80, 8--16.

\bibitem[Hertwig \& Pleskac(2010)]{hertwig2010decisions}
Hertwig, R., \& Pleskac, T.~J. (2010).
Decisions from experience: Why small samples?
\textit{Cognition}, 115(2), 225--237.

\bibitem[Nevo \& Erev(2012)]{nevo2012bi}
Nevo, I., \& Erev, I. (2012).
Bounded memory, inertia, sampling and weighting model for market entry games.
\textit{Games}, 3(1), 20--41.

\bibitem[Rand et al.(2014)]{rand2014social}
Rand, D.~G., Peysakhovich, A., Kraft-Todd, G.~T., Newman, G.~E., Wurzbacher, O., Nowak, M.~A., \& Greene, J.~D. (2014).
Social heuristics shape intuitive cooperation.
\textit{Nature Communications}, 5, 3677.

\bibitem[Rendell et al.(2010)]{rendell2010copy}
Rendell, L., Boyd, R., Cownden, D., et al. (2010).
Why copy others? Insights from the social learning strategies tournament.
\textit{Science}, 328(5975), 208--213.

\bibitem[Slovic(1993)]{slovic1993perceived}
Slovic, P. (1993).
Perceived risk, trust, and democracy.
\textit{Risk Analysis}, 13(6), 675--682.

\end{thebibliography}

\end{document}
